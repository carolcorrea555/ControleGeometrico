%%%%%%%%%%%%%%%%%%%%%%%%%%%%%%%%%%%%%%%%%
%%%%%%%% PROJETO MARCOS ALEXANDRINO
%%%%%%% TEMÁTICO DE PAOLO PICCIONE (2016)
%%%%%%%%%%%%%%%%%%%%%%%%%%%%%%%%%%%%%%%%%


\documentclass[11pt, reqno]{amsart}
%\documentclass[11pt]{article}

\usepackage[utf8]{inputenc}
\usepackage[brazil,english]{babel}
\usepackage{amsfonts,amsmath,amsthm}

\usepackage[margin=0.9in]{geometry}
%%
%%obs o BABEL nao interfere com outros pacotes. 
%Porem voce recebera 1 warning, referente a  ifenaçao do portugues
%ignore!

%%PACOTES DA CAROL

\usepackage[mathscr]{euscript}
\let\euscr\mathscr \let\mathscr\relax% just so we can load this and rsfs
\usepackage[scr]{rsfso}
\usepackage[utf8]{inputenc}
\usepackage{amsfonts}
\usepackage{stackengine}
\usepackage{amsmath}
\usepackage{amssymb}
\usepackage{amsthm}
\usepackage{enumitem}
\usepackage{mathtools}
\usepackage{bussproofs}
\usepackage{listings}
\usepackage{xcolor}
\usepackage[brazil]{babel}
\usepackage{geometry}
\usepackage{graphicx}
\usepackage{epstopdf}
\usepackage[hidelinks]{hyperref}
\usepackage{setspace}
\usepackage{nopageno}
\usepackage{fancyhdr}
\usepackage{tikz-cd}
\usepackage{float}

\usepackage{amsmath}
\usepackage{tikz}
\usetikzlibrary{fadings}
\usetikzlibrary{patterns}
\usetikzlibrary{shapes}

\newcommand{\mytitle}[1]{\textbf{\underline{#1}}}
\newcommand{\product}[3]{\displaystyle\prod_{#1}^#2 #3}
\newcommand{\gsum}[3]{\displaystyle\sum_{#1}^#2 #3}
\newcommand{\dint}[0]{\displaystyle\int}
\newcommand{\dsum}[0]{\displaystyle\sum}
\newcommand{\del}[0]{\partial}
\newcommand{\de}[0]{\text{d}}
\newcommand{\parti}[2]{\frac{\del #1}{\del #2}}
\newcommand{\deri}[2]{\frac{\de #1}{\de #2}}
\newcommand{\mdc}[0]{\text{mdc}}
\newcommand{\mmc}[0]{\text{mmc}}
\newcommand{\im}[0]{\text{Im}}
\newcommand{\e}[0]{\text{e}}
\newcommand{\ou}[0]{\text{ou}}
\newcommand{\se}[0]{\text{se}}
\newcommand{\tr}[0]{\text{tr}}
\newcommand{\<}[0]{\langle}
\newcommand{\Q}[0]{\mathbb{Q}}
\newcommand{\K}[0]{\mathbb{K}}
\renewcommand{\>}[0]{\rangle}
\newcommand{\ov}[0]{\overline}
\newcommand{\real}[0]{\mathbb{R}}
\newcommand{\rr}[0]{\mathbb{R}^2}
\newcommand{\rn}[0]{\mathbb{R}^n}
\newcommand\xrowht[2][0]{\addstackgap[.5\dimexpr#2\relax]{\vphantom{#1}}}
\newcommand{\card}[0]{\text{card}}
\newcommand{\aut}[0]{\text{Aut}}
\newcommand{\inn}[0]{\text{Inn}}
\newcommand{\ann}[0]{\text{Ann}}
\newcommand{\id}[0]{\text{Id}}
\newcommand{\into}[0]{\hookrightarrow}
\newcommand{\Z}[0]{\mathbb{Z}}
\newcommand{\T}[0]{\mathbb{T}}
\newcommand{\M}[0]{\mathbb{M}}
\renewcommand{\P}[0]{\mathbb{P}}
\renewcommand{\S}[0]{\mathbb{S}}
\newcommand{\D}[0]{\mathbb{D}}
\newcommand{\N}[0]{\mathbb{N}}
\newcommand{\disp}[0]{\displaystyle}
\newcommand{\mb}[1]{\mathbf{#1}}
\newcommand{\sgn}[0]{\text{sgn}}
\newcommand{\ab}[0]{\text{Ab}}
\newcommand{\Stab}[0]{\text{Stab}}
\newcommand{\Diff}[0]{\text{Diff}}
\newcommand{\Ad}[0]{\text{Ad}}
\newcommand{\ad}[0]{\text{ad}}
\newcommand{\Orb}[0]{\text{Orb}}
\newcommand{\grad}[0]{\text{grad}}
\newcommand{\can}[0]{\text{can}}
\renewcommand{\span}[0]{\text{span}}
\newcommand{\fr}[0]{\text{Fr}}
\newcommand{\Lie}[0]{\text{Lie}}
\newcommand{\Der}[0]{\text{Der}}
\newcommand{\Aut}[0]{\text{Aut}}
\newcommand{\X}[0]{\mathfrak{X}}
\newcommand{\eps}[0]{\varepsilon}
\newcommand{\prova}[0]{\textit{Prova: }}
\newcommand{\dem}[0]{\textit{Demonstração: }}

\newtheorem{lemma}{Lema}[section]
\newtheorem{definition}{Definição}[section]
\newtheorem{proposition}{Proposição}[section]
\newtheorem{example}{Exemplo}[section]
\newtheorem{theorem}{Teorema}[section]
\newtheorem{corollary}{Corolário}[section]

\counterwithin{equation}{section}
\counterwithin{figure}{section}

%% ACABOU OS PACOTES DA CAROL



\usepackage{hyperref}


%%%%%
%%%%%%% INICIO SEÇAO: MARCOS ALEXANDRINO
%%%%%

\begin{document}

\title[Geometric Aspects of Control Theory]{ \emph{Geometric Aspects of Control Theory}
\\ \footnotesize{(Monography)}}
\author[Carolina da Silva Correa, M. Alexandrino]{Candidate: Carolina da Silva Correa 
\\orientor: Prof Marcos M. Alexandrino}
%\date{20 de outubro 2016}
\maketitle


%\setcounter{tocdepth}{2}
\tableofcontents

\pagebreak


\part{Introduction}

\section{Motivation}

To introduce control systems, let's consider a very simple physical system:
a train moving along a 1-dimensional railway. If the train moves with
constant acceleration $a$, and at $t=0$ it's position and velocity
are $x_0$ and $v_0$ (respectively), it's motion is given by the solution
to the following ODE:

\begin{equation}
    \label{eq:train_cte}
    \begin{cases}
        x''(t) = a\\
        x(0) = x_0\\
        x'(0) = v_0
    \end{cases}
\end{equation}
\\

However, in reality, trains very rarely move with constant acceleration: usually,
theres a conductor controlling how the train moves. To consider a very simplified
model, let's say that, at any instant $t$, the conductor can choose
the acceleration $a(t)$ of the train, with the constraint $|a|\leq 1$.
We denote by $u: \real \to [-1,1]$
the function $t \mapsto a(t)$, which will be called the \textit{control function}.
The equation
of motion of the train now becomes:

\begin{equation}
    \label{eq:train_ctrl}
    \begin{cases}
        x''(t) = u(t)\\
        x(0) = x_0\\
        x'(0) = v_0
    \end{cases}
\end{equation}
\\

If $u: \real \to [-1,1]$ is well-behaved enough (see \autoref{apdx:EDOs}),
we can find a unique $x : I \subseteq \real \to \real$ that solves
this equation almost everywhere. This class of well-behaved
functions will be the \textit{admissible control functions}.
An important subset of the admissible control functions is the
class of \textit{piecewise constant function}, that is, functions
$u: \real \to [-1,1]$ such that there is a collection of intervals
covering $\real$ with $u$ being constant in the interior of each such interval.

\begin{figure}[H]
    \centering
    \includegraphics[scale = 0.5]{graf/piecewisecte.png}
    \includegraphics[scale=0.5]{graf/piecewisecte_sol.png}
    \caption{An exemple of a piecewise constant $u(t)$,
    and the corresponding solution $x(t)$ for
    $x_0=v_0=0$}
    \label{fig:piecewise_cte}
\end{figure}

The equation $x'' = u$ (with the constraint $|u| \leq 1$) is
an exemple of a \textit{control system} with \textit{control parameter}
$u$.
We will define precisely what this is later. For now, let's focus on what we
can do with a control system.

The possibility
of choosing the function $u$, and therefore the evolution
of the system represented by \autoref{eq:train_ctrl},
gives rise to many interesting problems, for exemple:

\begin{itemize}
    \item Given an initial state $(x_0, v_0)$, can we choose an admissible
    $u$ such that we eventually reach
    another state $(x_1, v_1)$? (If this is the case, we say that
    $(x_1, y_1)$ is reachable from $(x_0,v_0)$)

    \item Given an initial state $(x_0, v_0)$ and another state
    $(x_1, v_1)$ reachable from $(x_0,v_0)$,
    is there an admissible $u$ that minimizes
    the time we take to go from $(x_0,v_0)$ from $(x_1,v_1)$?
    If this is the case, how can we find such $u$?

    \item Given an initial state $(x_0, v_0)$ and another state
    $(x_1, v_1)$ reachable from $(x_0,v_0)$,
    is there an admissible $u$ that minimizes
    the amount of work we make to go from $(x_0,v_0)$ from $(x_1,v_1)$?
    If this is the case, how can we find such $u$?

\end{itemize}

In this simple exemple (and considering only piecewise constant controls),
all of those question can be answered using rather elementar
techiniques.

Let's now express our system in the language
of differential geometry to shed some
light on how we will approach control systems. 

We use
the usual trick to reduce the order of \autoref{eq:train_cte}
from $2$ to $1$. Consider the plane $\real^2$ with coordinates
$(y^1,y^2)$. Given $x : I \subseteq \real \to \real$ and $a \in \real$,
we define $q: I \to \real^2$ and $X_a:\real^2 \to \real^2$
by $q(t) = (x(t), x'(t))$, $X_a(y^1,y^2) = (y^2, a)$.
\autoref{eq:train_cte} can now be expressed as:

\begin{equation}
    \label{eq:train_field}
    \begin{cases}
        q'(t) = X_a(q(t))\\
        q(0) = (x_0, v_0) = q_0
    \end{cases}
\end{equation}

Notice that the solution to the equation above is completly determined by the vector
field $X_a \in \X(\real^2)$ and the initial condition
$q_0 \in \real^2$.

The control system expressed by the equation $x'' = u$
can now be expressed as $q' = X_u(q)$. The space
of possible values for the control parameter $u$, namely $U = [-1,1]$, is called
the \textit{space of control parameters}. Our
control system now can be sythesised as the (parametrized) family
of vector fields $\{X_u\}_{u \in U}$, each determining an ODE with which
our system can evolve.



\section{So... What is a control system?}


\part{Jáfizsla}


\section{Motivação}

\subsection{Campos Vetoriais e Fluxos}

Dado um campo vetorial $X \in \X(M)$,
podemos considerar o sistema dinâmico
dado pela seguinte equação diferencial:

$$\begin{cases}
    \gamma'(t) = X_{\gamma(t)}\\
    \gamma(0) = p
\end{cases}$$

onde $p \in M$ e $\gamma: I\subseteq \real \to M$ é uma curva (suave) em $M$.

Pelo teorema de existência e unicidade das soluções
de EDOs, cada $p \in M$ dá origem à uma única solução
maximal $\gamma_p : I_p \to M$ do sistema dinâmico.

Mais ainda, dado $t \in \real$,
sendo $M_t = \{p \in M : t \in I_p\}$, cada $M_t$ é
um aberto de $M$, e o mapa

$$P^t : M_t \to M_{-t}$$
$$p \mapsto \gamma_p(t)$$

é um difeomorfismo.

A família à $1$ parâmetro de difeomorfismos $t\mapsto P^t$ é dita o fluxo
de $X$, e vamos denotá-la por $P^t = e^{tX}$.

O campo vetorial $X$ é dito completo
se $M_t = M$ para cada $t \in \real$, ou equivalentemente
se $I_p = \real$ para cada $p \in M$. É notável que todo campo vetorial
com suporte compacto é completo.

\subsection{Sistemas de Controle}

No sistema dinâmico gerado por um único campo vetorial $X \in \X(M)$,
os estados futuros são completamente determinados pelo estado presente.
Mais especificamente, se o estado do sistema no tempo $t$ é dado
por $p(t)$, então $p(t) = e^{(t-t_0)X} (p(t_0))$ para qualquer $t_0$,
e em particular $p(t) = e^{tX} p(0)$.

Um sistema de controle geométrico
nada mais é que uma família de campos vetoriais
$\mathcal{F} \subseteq \X(M)$, onde interpretamos que podemos escolher,
controlar,
qual dos campos em $\mathcal{F}$ será utilizado para determinar o futuro
do sistema, e que podemos trocar a escolha de campo a qualquer momento.

É usual parametrizarmos a família $\mathcal{F}$ utilizando
algum conjunto $U$ (que a princípio não supomos possuir nenhuma estrutura
adicional),
escrevendo $\mathcal{F} = \{X_u\}_{u \in U}$. A variável $u$ é dita o parâmetro
de controle, e o conjunto $U$ o espaço de parâmetros de controle.

Uma função $u: \real \to U$ (chamada função controle) e um ponto $p \in M$ determinam
o seguinte sistema dinâmico:

$$\begin{cases}
    \gamma'(t) = (X_{u(t)})_{\gamma(t)}\\
    \gamma(0) = p
\end{cases}$$

A família à um parâmetro de campos $t \mapsto X_{u(t)}$
é dita um campo vetorial não-autônomo.
Se essa família for regular o suficiente (condições específicas são dadas na parte 4),
sempre existirá uma solução maximal $\gamma: I_p \to M$ absolutamente contínua
e que satisfaz
a equação diferencial em quase todo ponto (no sentido de medidas).

Um caso particular relevante é quando a função $u$ é constante por partes.
Nesse caso, dado um tempo $t \in \real$, existem $0 = T_0 < T_1< \dots< T_k = t$
tal que $u$ é constante
$(T_i, T_{i+1})$ para cada $0 \leq i < k$.
Sendo:

$$u((T_i,T_{i+1})) = \{u_{i+1}\}$$
$$t_{i+1} = T_{i+1} - T_i$$

A solução da equação diferencial é dada por:

$$\gamma(t) = e^{t_kX_{u_k}} \circ \dots \circ e^{t_1X_{u_1}} (p)$$

É bem comum (e é o que faremos pelo restante desse texto) tratarmos
apenas de funções controle constantes por partes.

Algo interessante
a ser estudado são quais pontos de $M$ podem ser alcançados
por um sistema de controle partindo de algum ponto inicial.
A propriedade de podermos chegar em qualquer
ponto final partindo de qualquer ponto inicial
é chamada controlabilidade.

Dado um sistema de controle $\mathcal{F} \subseteq \X(M)$ e um ponto
$p \in M$
o conjunto

$$\mathcal{A}_p = \{e^{t_kX_k} \circ \dots \circ e^{t_1X_1} (p) :
t_1, \dots, t_k > 0 ; X_1, \dots X_k \in \mathcal{F}\}$$

é chamado de conjunto alcançável do sistema de controle $\mathcal{F}$.

Relacionadas aos conjuntos alcançáveis,
e usualmente possuindo uma estrutura mais simples,
são as órbitas do sistema de controle.
Elas são os conjuntos da forma

$$\mathcal{O}_p = \{e^{t_kX_k} \circ \dots \circ e^{t_1X_1} (p) :
t_1, \dots, t_k \in \real ; X_1, \dots X_k \in \mathcal{F}\}$$

As órbitas de um sistema de controle podem ser vistas como
as órbitas da ação do menor pseudogrupo gerado pelos fluxos
dos campos em $\mathcal{F}$.

É notável que os conjuntos alcançáveis sempre são subconjuntos
das órbitas. Como as órbitas possuem uma estrutura bem simples
(de variedades imersas, como discutido na seção 5), estudá-las
nos permite construir uma boa base para o estudo dos conjuntos alcançáveis.

\section{Cálculo Cronológico}

Nesta sessão, desenvolveremos o chamado Cálculo Cronológico, que nos providenciará
uma notação muito útil para trabalharmos com familias de campos de vetores (ou seja,
sistemas de controle) e em particular será útilizada na prova do Teorema da Órbita.

O que faremos é criar um formalismo que nos permita tratar
o grupo de difeomorfismos $\Diff(M)$ como um grupo de Lie
(de dimensão infinita) com álgebra de Lie
$\X(M)$. Não faremos efetivamente isso, ou seja, não
providenciaremos $\Diff(M)$ com uma estrutura de variedade,
simplesmente desenvolveremos um formalismo que nos permita utilizar
notações análogas às de grupos de Lie.

Enunciaremos vários resultados sem demonstrá-los. As demonstrações
podem ser encontradas no capítulo 2 de \cite{Agrachev}.

\subsection{Estruturas em uma variedade em termos da álgebra $C^\infty(M)$}

Seja $M$ uma variedade, e denote por $C^\infty(M)$ a $\real$-álgebra
das funções suaves $M \to \real$, com soma e multiplicação definidas pontualmente:

$$ (a+b)(p) = a(p) + b(p)$$
$$ (a \cdot b)(p) = a(p)b(p)$$
$$ (\lambda \cdot a)(p) = \lambda a(p)$$

para $a,b \in C^\infty(M)$.

Vamos mostrar como podemos expressar algumas estruturas
de uma variedade (em particular, pontos, difeomorfismos e campos vetoriais)
em termos de $C^\infty(M)$.

Um ponto $p \in M$ define um homomorfismo de álgebras
$\hat{p} : C^\infty(M) \to \real$
dado pela avaliação $\hat{p}:f \mapsto f(p)$. Reciprocamente, temos que:

\begin{proposition}
    Dado um homomorfismo não-trivial de álgebras $\varphi: C^\infty(M) \to \real$,
    existe um ponto $p \in M$ tal que $\varphi = \hat{p}$
\end{proposition}

Podemos também reconstruir
a topologia e a estrutura suave de $M$ à partir de $C^\infty(M)$,
pois uma sequência de pontos
$p_i$ converge para $p$ se e somente se,
para todo $a \in C^\infty(M)$, $\hat{p}_i(a)$ converge para $\hat{p}(a)$,
e uma função $f:M \to \real$ é suave se e somente se
$f$ é da forma $p \mapsto \hat{p}(a)$ para algum $a \in C^\infty(M)$.

Similarmente, um difeomorfismo $P: M \to M$ define um isomorfismo
de álgebras $\hat{P} : C^\infty(M) \to C^\infty(M)$, dado pela composição
$\hat{P} : a \mapsto a \circ P$. Reciprocamente:

\begin{proposition}
    Dado um isomorfismo de álgebras $A: C^\infty(M) \to C^\infty(M)$,
    existe um difeomorfismo $P \in \Diff(M)$ tal que $A = \hat{P}$
\end{proposition}

Além disso, um vetor $X_p \in TM$ pode ser definido
como uma derivação pontual em $p$, $X_p: C^\infty(M) \to \real$
(ou seja, um mapa linear
tal que $X_p(ab) = a(p)X_p(b)  + X_p(a)b(p)$),
e dessa forma
um campo de vetores $X \in \X(M)$ define
uma derivação $\hat{X}: C^\infty(M) \to C^\infty(M)$ (ou seja, um mapa linear
tal que $\hat{X}(ab) = a\hat{X}(b)  + \hat{X}(a)b$), dado por
$\hat{X}(a)(p) = X_p a$. Reciprocamente,
dada uma derivação $D: C^\infty(M) \to C^\infty(M)$, temos um campo
de vetores $X$ tal que $D = \hat{X}$, dado por $X_p a = D(a)(p)$.

Portanto, a partir de agora, vamos identificar pontos de $M$ com homomorfismos
não-triviais
de álgebras $C^\infty(M) \to \real$, difeomorfismos de $M$
com
isomorfismos de álgebras $C^\infty(M) \to C^\infty(M)$,
e campos vetoriais de $M$ com derivações
$C^\infty(M) \to C^\infty(M)$.

Note que avaliar um campo $X$ num ponto $p$
é o mesmo que compor $p$ com $X$, isto é,
$X_p = p \circ X$. O mesmo vale para difeomorfismos
$P$, ou seja, $P(p) = p \circ P$.

\subsection{Topologia em $C^\infty(M)$}

Vamos definir uma topologia em $C^\infty(M)$ da seguinte maneira:

\begin{itemize}
    \item Escolha alguma imersão
    $M \to \real^N$, e sejam $h_i \in \X(M)$ a projeção
    ortogonal de $\parti{}{x^i} \in \X(\real^N)$ no espaço
    tangente à $M$, ponto à ponto.

    \item Dado $s \geq 0$, e $K \subseteq M$ um compacto, defina a seguinte seminorma:
    
    $$\|a\|_{s,K} = \sup\{|h_{i_1} \cdot \dots \circ h_{i_l} a (p)| :
    a \in C^\infty(M), p \in K, 0 \leq l \leq s\}$$

    \item Coloque em $C^\infty(M)$ a topologia gerada pelas seminormas
    $\|\|_{s,K}$
\end{itemize}

A topologia descrita acima não depende da escolha de imersão $M \to \real^N$,
e da à $C^\infty(M)$ a estrutura de um espaço de Frechét. É possível demonstrar
que todo campo $V \in \X(M)$ e todo difeomorfismo $P \in \Diff(M)$,
quando considerados como funcionais lineares em $C^\infty(M)$, são
contínuos.

\subsection{Familias à $1$ parâmetro de funcionais e operadores}

Como $C^\infty(M)$ é um espaço de Frechét,
dada uma familia à $1$ parâmetro $t \mapsto a_t \in C^\infty(M)$,
podemos falar de continuidade, diferenciabilidade e integrabilidade
dessas familias. Considere uma família $t \mapsto a_t$. Definimos
as seguinte propriedades:

\begin{itemize}
    \item Continuidade e diferenciabilidade da maneira usual em espaços de Frechét.
    
    \item $a_t$ é mensurável se, para cada $p \in M$,
    $t \mapsto a_t(p)$ é mensurável.

    \item $a_t$ é localmente integrável se ela é mensurável, e,
    para cada seminorma $\|\|_{s,K}$,
    $t_0, t_1 \in \real$, temos que:

    $$\int_{t_0}^{t_1} \|a_t\|_{s,K} dt < \infty$$

    Dada uma família $a_t$ localmente integrável e $t_0, t_1 \in \real$, podemos 
    definir a integral

    $$\int_{t_0}^{t_1} a_t dt \in C^{\infty}(M)$$

    que obedece as propriedades usuais de integrais

    \item $a_t$ é absolutamente contínua se existe uma família $b_t$ integrável
    tal que

    $$a_t = b_{t_0} + \int_{t_0}^t b_{t_0} dt$$

    \item $a_t$ é localmente limitada se para cada seminorma $\|\|_{s,K}$
    e cada intervalo compacto $I \subseteq \real$, existe uma constante
    $C_{s,K,I}$ tal que, para $t \in I$, $\|a_t\|_{s,K} \leq C_{s,K,I}$.
\end{itemize}

Dada uma família $t \mapsto A_t$
de funcionais (mapas $C^\infty(M) \to \real$) ou operadores
(mapas $C^\infty(M) \to C^\infty(M)$) lineares em $C^\infty(M)$,
dizemos que $A_t$ tem alguma propriedade (continuidade,
diferenciabilidade, integrabilidade, etc.)
se para cada $a \in C^\infty(M)$, $t \mapsto A_t a$ tem essa propriedade.

Mais ainda, definimos fracamente integrais e derivadas dessas famílias, ou seja,
sendo $A_t$ uma família de funcionais ou operadores lineares em $C^\infty(M)$,
definimos:

\begin{itemize}
    \item Se $A_t$ é diferenciável,
    definimos 
    
    $$\deri{}{t}A_t : a \mapsto \deri{}{t}(A_ta)$$

    \item Se $A_t$ é localmente integrável, definimos
    
    $$\int_{t_0}^{t_1} A_t dt : a \mapsto \int_{t_0}^{t_1} (A_ta) dt$$
\end{itemize}

Integrais e derivadas de funcionais e operadores lineares também são funcionais ou operadores
lineares, pela linearidade de integrais e derivadas.

Além disso, derivadas de familias de operadores obedecem a regra de Leibniz, isto é:

$$\deri{}{t} A_t \circ B_t = \deri{}{t} A_t \circ B_t + A_t \circ \deri{}{t}B_t$$

\subsection{Campos não-autônomos e a exponencial cronológica} 

Um campo vetorial não autônomo é uma família à $1$ parâmetro
$t \mapsto X_t$ de campos vetoriais (ou seja, $X_t \in \X(M)$)
que seja localmente limitada.
Dado um campo vetorial não autônomo, podemos considerar a EDO:

$$\begin{cases}
    \gamma'(t) = \gamma(t) \circ X_t\\
    \gamma(0) = p_0
\end{cases}$$

Utilizando o teorema de Caratheodory para soluções de EDOs e
o fato de $X_t$ ser localmente limitado, podemos demonstrar
que existe uma solução maximal $\gamma_{p_0} : I_{p_0} \to M$
Lipchitz contínua e diferenciável para quase todo $t \in I_{p_0}$
que satisfaz a EDO para quase todo $t \in I_{p_0}$.

Mais ainda, podemos demonstrar que $p \mapsto \gamma_{p}(t)$ é suave
para cada $t \in \real$ fixo, e portanto,
a família $P^t: p \mapsto \gamma_p(t)$ é uma família a um parâmetro de difeomorfismo
de $M$. Denotamos

$$P^t = \overrightarrow{\exp} \int_0^t X_\tau d\tau$$,

a exponecial cronológica pela direita.

Ela é a solução única da seguinte equação diferencial:

$$\begin{cases}
    \dot{P}^t = P^t \circ X_t\\
    P^0 = \id
\end{cases}$$

Considere o inverso $Q^t = (P^t)^{-1}$. Ele satisfaz $P^t \circ Q^t = \id$, e portanto:

$$\dot{P}^t \circ Q^t + P^t \circ \dot{Q}^t = 0 \implies P^t \circ X_t \circ Q^t + P^t \circ \dot{Q}^t
\implies \dot{Q}^t = (-X_t) \circ Q^t$$

Dessa forma denotamos:

$$Q^t = \overleftarrow{\exp} \int_0^t (-X_\tau) d\tau$$

a exponecial cronológica pela esquerda.

O fluxo $e^{tX}$ de um campo vetorial é um caso específico da
exponecial cronológica,
mais especificamente quando $X_t$ é a constante $X$.
Ele satisfaz:

$$\deri{}{t} e^{tX} = X \circ e^{tX} = e^{tX} \circ X$$

\subsection{O colchete de Lie e a Adjunta}

Considere dois campos $X, Y \in \X(M)$. A composição $X \circ Y$ não é um
campo (ou seja, uma derivação de $C^\infty(M)$):

$$(X \circ Y) (ab) = X(Y(a)b + aY(b)) = (X\circ Y)(a)b + Y(a)X(b) + X(a)Y(b) + a(X\circ Y)(b)$$

No entando, é facil verificar que $X\circ Y - Y \circ X$ é um campo:

$$(X\circ Y - Y \circ X)(ab) = (X\circ Y - Y \circ X)(a)b
+ a(X\circ Y - Y \circ X)(b)$$

Denotamos esse campo por:

$$[X,Y] = X\circ Y - Y \circ X$$

o colchete de Lie de $X$ e $Y$. Ele possui as seguintes propriedades:

\begin{itemize}
    \item (\textbf{Anti-comutatividade}) $[X,Y] = -[Y,X]$
    \item (\textbf{Identidade de Jacobi/Regra de Leibniz}) $[X,[Y,Z]]=
    [[X,Y],Z] + [Y,[X,Z]]$  
\end{itemize}

E portanto $\X(M)$ com o colchete de Lie forma uma álgebra de Lie.

O significado geométrico do colchete de Lie
pode ser melhor entendido utilizando pullbacksde campos vetoriais por
difeomorfismos. Para fazermos isso, vamos primeiro
expressá-los usando a linguagem do cálculo cronológico.

Dessa forma, seja $X_p \in T_p M$, $P \in \Diff(M)$, e $\gamma: I \to M$ uma curva
com $\gamma'(0) = X_p$. Seja ainda $a \in C^\infty(M)$. Note que:

$$(dP_p X_p) a = (\deri{}{t}\bigg|_{t=0} \gamma(t) \circ P) a = 
((\deri{}{t}\bigg|_{t=0}\gamma(t)) \circ P) a = (X_p \circ P)a$$

E portanto $dP_p X_p = X_p \circ P$.

O pullback de um campo vetorial $X \in \X(M)$ por $P \in \Diff(M)$ é dado por:

$$p \circ (P^* X) = d(P^{-1})_{P(p)} X_{P(p)} =  p \circ P \circ X \circ P^{-1}$$.

E portanto $(P^* X) = P \circ X \circ P^{-1}$.
Denotamos:

$$(\Ad P) X = P \circ X \circ P^{-1}$$

É facilmente verificável que $\Ad P:\X(M) \to \X(M)$ é um automorfismo de
álgebras de Lie.

Chamamos $\Ad : \Diff(M) \to \Aut(\X(M)))$ de representação adjunta,
pois possui propriedades bastante similares à representação adjunta de
grupos de Lie. Da mesma maneira que em grupos de Lie, denotamos:

$$(\ad X) Y = \deri{}{t}\bigg|_{t=0} (\Ad e^{tX}) Y$$

Verificarmos que:

$$(\ad X) Y = \deri{}{t}\bigg|_{t=0}
(e^{tX} \circ Y \circ e^{-tX}) = X \circ Y - Y \circ X = [X,Y]$$

$\ad : \X(M) \to \Der(\X(M))$ é a representação adjunta da álgebra de Lie $\X(M)$ (onde
$\Der(\X(M))$ denota as derivações de $\X(M)$).

\section{O Teorema da Órbita}

Nessa seção discutiremos as órbitas de um sistema de controle
e alguns conceitos relacionados (sistemas de controle localmente finitamente gerados,
integração de distribuições singulares e não-singulares).
Para simplificarmos a notação, consideraremos que todos
os campos são completos, mas a discussão a seguir pode ser facilmente
traduzida para campos não completos utilizando formalismos
de pseudogrupos e feixes.

O teorema que descreve a estrutura das órbitas é o seguinte:

\begin{theorem}[Teorema da Órbita/Teorema de Sussmann]
    Seja $\mathcal{F} \subseteq \X(M)$ e $p_0 \in M$. Então:

    \begin{itemize}
        \item $\mathcal{O}_{p_0}$ é uma subvariedade imersa e conexa de $M$
        \item $T_p \mathcal{O}_{p_0} = \span\{ p  \circ (\Ad\ P) X : P \in \mathcal{P}, X \in \mathcal{F}\}$
    \end{itemize}
\end{theorem}

Onde $\mathcal{P}$ é o grupo gerado pelos fluxos dos campos em $\mathcal{F}$. Explicitamente:

$$\mathcal{P} = \{e^{t_1X_1} \circ \dots \circ e^{t_kX_k}:
t_1, \dots, t_k \in \real ; X_1, \dots X_k \in \mathcal{F}\}$$

\subsection{Demonstração do Teorema de Sussmann}

Primeiramente, imagine que $\mathcal{O}_{p_0}$ é, de fato, uma variedade imersa,
e seja $p \in \mathcal{O}_{p_0}$.
Para cada $P \in \mathcal{P}$, $X \in \mathcal{F}$,
$t \mapsto p \circ P \circ e^{tX} \circ P^{-1}$ é uma curva
em $\mathcal{O}_{p_0}$, e portanto $p \circ (\Ad P) X \in T_p \mathcal{O}_{p_0}$
(como no esquema abaixo).\\



\tikzset{every picture/.style={line width=0.75pt}} %set default line width to 0.75pt        
\begin{tikzpicture}[x=0.75pt,y=0.75pt,yscale=-1,xscale=1]
%uncomment if require: \path (0,255); %set diagram left start at 0, and has height of 255

%Shape: Free Drawing [id:dp49245771404128114] 
\draw  [color={rgb, 255:red, 0; green, 0; blue, 0 }  ][line width=3] [line join = round][line cap = round] (316.8,160) .. controls (316.8,160) and (316.8,160) .. (316.8,160) ;
%Shape: Free Drawing [id:dp8730283138166564] 
\draw  [color={rgb, 255:red, 0; green, 0; blue, 0 }  ][line width=3] [line join = round][line cap = round] (401.8,72) .. controls (402.13,72) and (402.47,72) .. (402.8,72) ;
%Straight Lines [id:da0031941210242911744] 
\draw [color={rgb, 255:red, 144; green, 19; blue, 254 }  ,draw opacity=1 ]   (317,161) -- (353.31,193.66) ;
\draw [shift={(354.8,195)}, rotate = 221.97] [color={rgb, 255:red, 144; green, 19; blue, 254 }  ,draw opacity=1 ][line width=0.75]    (10.93,-3.29) .. controls (6.95,-1.4) and (3.31,-0.3) .. (0,0) .. controls (3.31,0.3) and (6.95,1.4) .. (10.93,3.29)   ;
%Straight Lines [id:da5503181757486022] 
\draw [color={rgb, 255:red, 144; green, 19; blue, 254 }  ,draw opacity=1 ]   (402,72) -- (438.31,104.66) ;
\draw [shift={(439.8,106)}, rotate = 221.97] [color={rgb, 255:red, 144; green, 19; blue, 254 }  ,draw opacity=1 ][line width=0.75]    (10.93,-3.29) .. controls (6.95,-1.4) and (3.31,-0.3) .. (0,0) .. controls (3.31,0.3) and (6.95,1.4) .. (10.93,3.29)   ;
%Curve Lines [id:da27326141828123984] 
\draw [color={rgb, 255:red, 208; green, 2; blue, 27 }  ,draw opacity=1 ]   (268.8,150) .. controls (308.8,120) and (304.8,186) .. (391.8,189) ;
%Curve Lines [id:da863373974756469] 
\draw [color={rgb, 255:red, 208; green, 2; blue, 27 }  ,draw opacity=1 ]   (355.8,64) .. controls (395.8,34) and (391.8,100) .. (478.8,103) ;

% Text Node
\draw (314,135.4) node [anchor=north west][inner sep=0.75pt]    {$p$};
% Text Node
\draw (392,46.4) node [anchor=north west][inner sep=0.75pt]    {$p\circ P$};
% Text Node
\draw (331,198.4) node [anchor=north west][inner sep=0.75pt]  [color={rgb, 255:red, 144; green, 19; blue, 254 }  ,opacity=1 ]  {$p\circ \left(\left(\text{Ad} \ P\right) X\right)$};
% Text Node
\draw (404,110.4) node [anchor=north west][inner sep=0.75pt]  [color={rgb, 255:red, 144; green, 19; blue, 254 }  ,opacity=1 ]  {$p\circ P\circ X$};
% Text Node
\draw (328,33.4) node [anchor=north west][inner sep=0.75pt]  [color={rgb, 255:red, 208; green, 2; blue, 27 }  ,opacity=1 ]  {$p\circ P\circ e^{tX}$};
% Text Node
\draw (214,117.4) node [anchor=north west][inner sep=0.75pt]  [color={rgb, 255:red, 208; green, 2; blue, 27 }  ,opacity=1 ]  {$p\circ P\circ e^{tX} \circ P^{-1}$};


\end{tikzpicture}\\

Dessa forma, vamos definir:

$$\Pi_p = \span \{ p  \circ (\Ad\ P) X : P \in \mathcal{P}, X \in \mathcal{F}\}$$

Esse será nosso candidato à espaço tangente $T_p \mathcal{O}_{p_0}$.

\begin{lemma}
    Para cada $p \in \mathcal{O}_{p_0}$, $\dim \Pi_p = \dim \Pi_{p_0}$
\end{lemma}

\dem Necessariamente existe $Q \in \mathcal{P}$ tal que $p = p_0 \circ Q$.
Considere o isomorfismo $dQ_{p_0}:T_{p_0}M \to T_p M$. Dado um elemento
arbitrário $p_0 \circ (\Ad P) X \in \Pi_{p_0}$, temos que:

$$dQ_{p_0}(p_0 \circ (\Ad P) X \in \Pi_{p_0}) = 
p_0 \circ P \circ X \circ P^{-1} \circ Q =$$
$$= p_0 \circ Q \circ Q^{-1} \circ P \circ X
\circ P^{-1} \circ Q = (p_0 \circ Q) \circ (\Ad(Q^{-1} \circ P) X) \in \Pi_p$$

E dessa forma, $dQ_{p_0}(\Pi_{p_0}) \subseteq \Pi_p$. Similarmente,
$d(Q^{-1})_p(\Pi_p) \subseteq \Pi_{p_0}$. Portanto, $\Pi_p$ e $\Pi_{p_0}$
tem a mesma dimensão. \qed\\

Introduzimos a notação:

$$(\Ad \mathcal{P})\mathcal{F} = \{(\Ad\ P) X : P \in \mathcal{P}, X \in \mathcal{F}\}$$\\

Vamos agora colocar uma topologia e estrutura suave em $\mathcal{O}_{p_0}$.

Seja $m = \dim \Pi_{p_0}$. Para um ponto arbitrário $p \in \mathcal{O}_{p_0}$,
sejam $V_1, \dots, V_m \in (\Ad \mathcal{P})\mathcal{F}$ tais que
$\{p \circ V_1, \dots, p \circ V_m\}$ seja uma base de $\Pi_p$. Introdizimos o mapa:

$$\psi_p : (t_1, \dots, t_m) \mapsto p \circ e^{t_1V_1} \circ \dots \circ e^{t_mV_m}$$

Primeiramente, demonstramos que a imagem de $\psi_p$ está contida em $\mathcal{O}_{p_0}$.
De fato, cada $V_i$ pode ser escrito como $V_i = (\Ad P_i) X_i$ para $P_i \in \mathcal{P}$
e $X_i \in \mathcal{F}$. Dessa forma:

$$e^{tV_i} = e^{t (\Ad P_i) X_i} = P_i \circ e^{t X_i} \circ P_i^{-1} \in \mathcal{P}$$

e portanto $\psi_p(t_1,\dots,t_m) \in \mathcal{O}_{p_0}$.

Como

$$\parti{\psi_p}{t_i}\bigg|_0 = q \circ V_i$$

$\psi_p|_O$ é uma imersão para uma vizinhaça $O \subseteq \real^m$ suficiente pequena
da origem.

Os conjuntos da forma $\psi_p(O)$,
onde $O$ é uma vizinhaça da origem tal que
$\psi_p|_O$ é uma imersão, são candidatos
à base de uma topologia em $\mathcal{O}_{p_0}$. Vamos demonstrar
algumas propriedades desses conjuntos:

\begin{itemize}
    \item Para $t \in O$, $(d\psi_p)_t(T_t \real^m) = \Pi_{\psi_p(t)}$.
    
    Como o posto de $\psi_p|_O$ é $m$ e $\dim \Pi_{\psi_p(t)} = m$, basta demonstrarmos
    que $\parti{\psi_p}{t_i}\bigg|_t \in \Pi_{\psi_p(t)}$. Temos que:

    $$\parti{\psi_p}{t_i}\bigg|_t =
    \parti{}{t_i}\bigg|_t p \circ e^{t_1V_1} \circ \dots \circ e^{t_mV_m} =
    p \circ e^{t_1V_1} \circ \dots\circ e^{t_iV_i}\circ V_i \circ e^{t_{i+1}V_{i+1}}\circ 
    \dots \circ e^{t_mV_m}=$$

    $$=p \circ e^{t_1V_1} \circ
    \dots\circ e^{t_mV_m} \circ e^{-t_m V_m}\circ \dots \circ e^{-t_{i+1}V_{i+1}}
    \circ V_i \circ e^{t_{i+1}V_{i+1}}\circ 
    \dots \circ e^{t_mV_m}$$

    Sendo $Q = e^{t_{i+1}V_{i+1}}\circ 
    \dots \circ e^{t_mV_m}$, temos:

    $$\parti{\psi_p}{t_i}\bigg|_t = \psi_p(t) \circ (\Ad Q^{-1})V_i \in \Pi_{\psi_p(t)}$$ \qed

    \item Os conjuntos da forma $\psi_p(O)$ formam uma base para uma topologia
    em $\mathcal{O}_{p_0}$. O espaço topológico gerado por essa base será denotado
    por $\mathcal{O}_{p_0}^\mathcal{F}$
    
    É sufiente demonstrarmos que dado $\psi_p(O)$
    e $p' \in \psi_p(O)$, para $O'$ pequeno o suficiente,
    $\psi_{p'}(O') \subseteq \psi_p(O)$

    Sejam $V_1', \dots, V_m'\in \mathcal{F}$ os campos tais que

    $$\psi_{p'}(t_1,\dots,t_m) = p' \circ e^{t_1V_1'} \circ \dots \circ e^{t_mV_m'}$$

    Considere primeiramente a curva $t_1 \mapsto p' \circ e^{t_1V_1'}$. Como
    para $t_1$ pequeno o suficiente
    sua velocidade $(p' \circ e^{t_1V_1'}) \circ V_1'$ pertence à
    $T_{p' \circ e^{t_1V_1'}} \psi_{p'}(O') =
    \Pi_{p' \circ e^{t_1V_1'}} = T_{p' \circ e^{t_1V_1'}} \psi_{p}(O)$,
    para $t_1$ pequeno o sufiente $(p' \circ e^{t_1V_1'}) \in \psi_{p}(O)$.

    Aplicando o mesmo argumento à curva
    $t_2 \mapsto p' \circ e^{t_1V_1'} \circ e^{t_2V_2'}$,
    obtemos que $p' \circ e^{t_1V_1'} \circ e^{t_2V_2'} \in \psi_{p}(O)$ para
    $t_1$ e $t_2$ pequenos o suficiente, e prosseguindo indutivamente,
    obtemos que $\psi_{p'}(t) \in \psi_{p}(O)$ para $t$ pequeno o suficiente. \qed

    \item A espaço $\mathcal{O}_{p_0}^\mathcal{F}$ é conexo.
    
    Basta notarmos que os mapas $t \mapsto p \circ e^{tX}$ são contínuos
    em$\mathcal{O}_{p_0}^\mathcal{F}$
    para $X \in \mathcal{F}$, e portanto quaisquer dois pontos de
    $\mathcal{O}_{p_0}^\mathcal{F}$ podem ser conectados por curvas contínuas
    e portanto $\mathcal{O}_{p_0}^\mathcal{F}$ é conexo.

\end{itemize}

Induzimos agora uma estrutura suave em $\mathcal{O}_{p_0}^\mathcal{F}$ declarando
os mapas $\psi_p|_O$ como cartas. Note que $T_p \mathcal{O}_{p_0}^\mathcal{F} = \Pi_p$.


Isso conclui a demonstração do Teorema de Órbita. \qed

\subsection{Folheações}
Considere uma partição $L = \{L_\alpha\}$ de $M$ em variedades conexas imersas.
As variedades imersas $L_\alpha$ são chamadas de folhas da partição,
e a folha que contém um ponto $p \in M$ é escrita como $L_p$.

\begin{definition}
    Uma partição $\{L_\alpha\}$ de $M$ em variedades conexas imersas
    é dita uma folheção singular se para cada $p \in M$ e cada vetor
    $v \in T_p L_p$, existe um campo $X \in \X(M)$ tangente as folhas da partição
    tal que $X_p = v$.
\end{definition}

\begin{definition}
    Uma folheação singular $\{L_\alpha\}$ é dita regular
    se todas as suas folhas possuem a mesma dimensão.
\end{definition}

\begin{theorem}
    As órbitas $\{\mathcal{O}_p\}$ de um sistema
    de controle $\mathcal{F}$ formam uma folheação singular.
\end{theorem}

\dem Sejam $V_1, \dots, V_m \in (\Ad \mathcal{P})\mathcal{F}$ tais que
$\{p \circ V_1, \dots, p \circ V_m\}$ seja uma base de $T_p\mathcal{O}_p$.
Então, qualquer $v \in T_p\mathcal{O}_p$ pode ser escrito como
$v = \dsum t_i (p \circ V_i)$.

Simplesmente defina o campo $X = \dsum t_i V_i$. Como cada
$V_i \in (\Ad \mathcal{P})\mathcal{F}$, para cada $q \in M$,
$X_q = \dsum t_i (q \circ V_i)\in T_q\mathcal{O}_q$. \qed

Dada uma folheação singular $L$, indicamos por $\X(L)$ o conjunto dos campos
em $M$ tangentes à folheação, ou seja:

$$\X(L) = \{X \in \X(M) : \forall p \in M, X_p \in T_pL_p\}$$

Podemos facilmente verificar, pontualmente, que $\X(L)$ é um $C^\infty(M)$-submódulo
de $\X(M)$:

\begin{proposition}
    $\X(L)$ é um $C^\infty(M)$-submódulo
    de $\X(M)$
\end{proposition}

\dem Sejam $X, Y \in \X(L)$, $f \in C^\infty(M)$. Temos que:

\begin{itemize}
    \item $(X+Y)_p = X_p + Y_p \in T_pL_p$
    \item $(fX)_p = f(p)X_p \in T_pL_p$
\end{itemize} \qed

Além disso, podemos demonstrar:

\begin{proposition}
    $\X(L)$ é uma subálgebra de Lie de $\X(M)$
\end{proposition}

\dem Tendo em vista a proposição anterior, basta demonstrarmos
que dados $X, Y \in \X(L)$, $[X,Y] \in \X(L)$.

Considere a família à $1$ parâmetro $t \mapsto (\Ad e^{tX}) Y$. Como
$e^{tX}$ preserva a folha onde estão os pontos (isto é,
$p \in L_\alpha \iff p\circ e^{tX} \in L_\alpha$),
para cada $p \in M$, $p \circ (\Ad e^{tX}) Y \in T_pL_p$,
e portanto:

$$\deri{}{t}\bigg|_{t=0} p \circ (\Ad e^{tX}) Y = p \circ (\ad X) Y = p \circ [X,Y] \in T_pL_p$$

Ou seja, $[X,Y] \in \X(L)$ \qed

Com isso, podemos fazer uma estimativa inferior do espaço tangente às órbitas
de um sistema de controle:

\begin{definition}
    Seja $\mathcal{F} \subseteq \X(M)$. Definimos
    $\Lie \mathcal{F}$ como sendo a menor subálgebra de Lie
    de $\X(M)$ que contenha $\mathcal{F}$ e também seja
    um $C^\infty(M)$-submódulo de $\X(M)$.
\end{definition}

Note que a definição acima está bem definida pois a intersecção
de uma família de subálgebras de Lie que também são $C^\infty(M)$-submódulos
de $\X(M)$ também é uma subálgebra de Lie e um $C^\infty(M)$-submódulo.

\begin{definition}
    $\Lie_p \mathcal{F} = \{X_p : X \in \Lie\mathcal{F}\} $
\end{definition}

Indicando por $\mathcal{O}$ a folheção de $M$ nas órbitas
de $\mathcal{F}$, é claro que $\Lie \mathcal{F} \subseteq \X(\mathcal{O})$.
Dessa forma, temos:

\begin{proposition}
    $\Lie_p\mathcal{F} \subseteq T_p\mathcal{O}_p$
\end{proposition}

\subsection{Uma melhor descrição de $T_p \mathcal{O}_p$} Existe um caso específico,
no qual famílias da campos analíticos são inclusas, onde podemos melhor descrever
o espaço tangente à uma órbita $T_p\mathcal{O}_p$. É o
caso das famílias localmente finitamente geradas. Vamos agora explorar este caso.

\begin{definition}
    Um $C^\infty(M)$-submódulo $\mathcal{V}\subseteq \X(M)$ é dito finitamente
    gerado se exitem campos $V_1,\dots,V_k$ tal que:

    $$\mathcal{V} = \{\sum a_i V_i : a_i \in C^\infty(M)\}$$

    O conjunto $V_1,\dots,V_k$ é dito um gerador de $\mathcal{V}$.
\end{definition}

\begin{proposition}
    Seja $\mathcal{V}\subseteq \X(M)$ um $C^\infty(M)$-submódulo
    finitamente gerado, e $X \in \X(M)$. Então,
    se para todo $V \in \mathcal{V}$, temos que:

    $$(\ad X)V \in \mathcal{V}$$

    Segue que:

    $$(\Ad e^{tX})V \in \mathcal{V}$$

    Para todo $V \in \mathcal{V}$, $t \in \real$.
\end{proposition}

\dem Seja $V_1, \dots, V_k$ um conjunto gerador de $\mathcal{V}$. Por hipótese, temos:

$$[X,V_i] = \sum a_{ij} V_j$$

É suficiente mostrarmos que para cada $1 \leq i \leq k$ e cada $t \in \real$,
temos que:

$$V_i(t) = (\Ad e^{tX}) V_i \in \mathcal{V}$$

Diferenciando $V_i(t)$, obtemos a seguinte equação diferencial:

$$\dot{V_i}(t) = (\Ad e^{tX}) [X,V_i] = \sum (\Ad e^{tX}) (a_{ij} V_j)
= \sum (e^{tX} a_{ij}) V_j(t)$$

Definindo $a_{ij}(t) = e^{tX} a_{ij}$, e avaliando num ponto $p \in M$:

$$\dot{(p\circ V_i)}(t) = \sum p(a_{ij}(t))(p \circ V_j(t))$$

Defina a matriz $A_p(t) = [p(a_{ij}(t))]$. Se $\Gamma_p(t) = [\gamma_{p,ij}(t)]$
é a solução da EDO:

$$\dot{\Gamma}_p = A_p(t) \Gamma_p$$
$$\Gamma_p(0) = \id$$

Então, como $p \mapsto A_p(t)$ é suave, $p \mapsto \Gamma_p(t)$ também é suave.
Ou seja, as funções $\gamma_{ij} : p \mapsto \gamma_{p,ij}$ são suaves.

Dessa forma, $V_i(t)$ pode ser escrito como:

$$V_i(t) = \sum \gamma_{ij}(t) V_i(0) = \sum \gamma_{ij} V_i \in \mathcal{V}$$ \qed

Vamos introduzir agora submódulos localmente finitamente gerados.

\begin{definition}
    Dada uma família $\mathcal{F} \subseteq \X(M)$, e $U \subseteq M$ um aberto,
    definimos:

    $$\mathcal{F}|_U = \{X|_U : X \in \mathcal{F}\} \subseteq \X(U)$$
\end{definition}

\begin{definition}
    Um $C^\infty(M)$-submódulo $\mathcal{V}\subseteq \X(M)$ é dito localmente
    finitamente
    gerado se para cada $p \in M$, existe uma vizinhaça aberta $U$ de $p$
    tal que $\mathcal{V}|_U$ é finitamente gerado.
\end{definition}

E demonstramos um análogo à proposição 4.3 para submódulos localmente finitamente gerados:

\begin{proposition}
    Seja $\mathcal{V}\subseteq \X(M)$ um $C^\infty(M)$-submódulo
    localmente finitamente gerado, e $X \in \X(M)$. Então,
    se para todo $V \in \mathcal{V}$, temos que:

    $$(\ad X)V \in \mathcal{V}$$

    Segue que:

    $$(\Ad e^{tX})V \in \mathcal{V}$$

    Para todo $V \in \mathcal{V}$, $t \in \real$.
\end{proposition}

\dem Seja $p \in M$, e $U$ uma vizinhaça de $p$ tal que $\mathcal{V}|_U$ é finitamente
gerado. Temos que, para cada $V \in \mathcal{V}$, $(\ad X|_U) V_U \in \mathcal{V}|_U$,
e portanto $(\Ad e^{tX|_U})V|_U \in \mathcal{V}|_U$. Como $p$ é arbitrário,
temos que $(\Ad e^{tX})V \in \mathcal{V}$ \qed

Por fim, estamos prontos para demonstrar:

\begin{theorem}
    Seja $\mathcal{F} \subseteq \X(M)$ uma família tal que $\Lie \mathcal{F}$ seja
    localmente finitamente gerada. Então:

    $$T_p\mathcal{O}_p = \Lie_p \mathcal{F}$$
\end{theorem}

\dem Note que por definição, para todo $X \in \mathcal{F}$, $Y \in \Lie \mathcal{F}$,
$(\ad X) Y \in \Lie \mathcal{F}$.

Dessa forma, $(\Ad e^{tX}) Y \in \Lie \mathcal{F}$.

Como

$$T_p\mathcal{O}_p= \span\{(\Ad e^{t_1X_1}\circ \dots \circ e^{t_kX_k}) Y : X_1,\dots,X_k, Y \in \mathcal{F} ; t_1,\dots,t_k \in \real\}$$,

temos que $T_p\mathcal{O}_p \subseteq \Lie_p\mathcal{F}$. Como
$\Lie_p\mathcal{F} \subseteq T_p\mathcal{O}_p$, temos que $\Lie_p\mathcal{F} = T_p\mathcal{O}_p$. \qed



\subsection{Integração de Distribuições} Como aplicação do Teorema da Órbita, derivamos
alguns teoremas para integração de distribuições: o Teorema de Frobenius (para distribuições regulares)
e o Teorema de Stefan-Sussmann (para distribuições singulares).

\begin{definition}
    Uma distribuição singular $\Delta$ consiste da escolha de um subespaço
$\Delta_p \subseteq T_p M$ para cada $p \in M$, de forma que para cada $p \in M$,
exista uma vizinhaça $U_p$ de $p$ e uma coleção de campos $X_1, \dots, X_k$
tal que, para $q \in U_p$, $\Delta_q = \span\{q\circ X_1,\dots,q \circ X_k\}$.
\end{definition}

\begin{definition}
    Uma distribuição singular $\Delta$ é dita regular de posto $k$ se
    cada $\Delta_p$ tem dimensão $k$.
\end{definition}

Dizemos que uma distribuição (singular) $\Delta$ é integrável
se existe uma folheação (singular) $L$ tal que, para todo
$p \in M$, $T_p L_p = \Delta_p$. Cada $L_p$ é dita uma variedade integral
de $\Delta$, e $L$ é dita uma folheação integral de $\Delta$.

Note que uma distribuição regular integrável sempre será integrada em uma
folheação regular. Existem exemplos bem simples de distribuições não integráveis
em dimensão baixa.

De maneira similar à folheações, indicamos por $\X(\Delta)$ os campos tangentes à
uma distribuição $\Delta$:

$$\X(\Delta) = \{X \in \X(M) : \forall p \in M, X_p \in \Delta_p\}$$

Ainda similarmente às folheações,
podemos facilmente verificar, pontualmente, que $\X(\Delta)$
é um $C^\infty(M)$-submódulo
de $\X(M)$:

\begin{proposition}
    $\X(\Delta)$ é um $C^\infty(M)$-submódulo
    de $\X(M)$
\end{proposition}

\dem Dados, $X,Y \in \X(\Delta)$ e $f \in C^\infty(M)$, basta verificarmos pontualmente:

\begin{itemize}
    \item $(X+Y)_p = X_p + Y_p \in \Delta_p$
    \item $(fX)_p = f(p) X_p \in \Delta_p$
\end{itemize} \qed

No entanto, diferentemente das folheações, $\X(\Delta)$ pode não ser
uma sub-álgebra de Lie de $\X(M)$.

\begin{definition}
    Uma distribuição $\Delta$ satisfaz a condição de Frobenius
    se $\X(\Delta)$ é uma sub-álgebra de Lie de $\X(M)$.
\end{definition}

Note que, se $\Delta$ é integrável, e $L$ é a folheação integral de $\Delta$,
então $\X(\Delta) = \X(L)$. Dessa forma, a condição de Frobenius é uma condição
necessária para a integrabilidade de uma distribuição.

O clássico Teorema de Frobenius diz que ela também é suficiente no caso regular:

\begin{theorem}[Teorema de Frobenius]
    Seja $\Delta$ uma distribuição regular. Então, $\Delta$ é integrável
    se e somente se $\Delta$ satisfaz a condição de Frobenius.
\end{theorem}

\dem Já vimos que se $\Delta$ é integrável então $\Delta$
satisfaz a condição de Frobenius.
Suponha que $\Delta$ satisfaz a condição de Frobenius.
Considere a família $\X(\Delta) \subseteq \X(M)$ como um sistema de controle
geométrico.
Então, $\Lie \X(\Delta) = \X(\Delta)$, e dessa forma
para cada $p \in M$, $\Lie_p \X(\Delta) = \Delta_p$.

Além disso, podemos demonstrar que $\X(\Delta)$ é localmente
finitamente gerada: dado $p \in M$ e $V \in \X(\Delta)$, seja $U$ uma vizinhaça
de $p$ tal que existam campos $X_1, \dots, X_k$ onde, para $q \in U$,
$\Delta_q = \span\{q\circ X_1, \dots, q\circ X_k\}$. Para cada $q\in U$,
existe únicos (são únicos
pois cada $\Delta_q$ tem dimensão $k$,
e temos que $k$ vetores que geram $\Delta_q$, ou seja,
eles formam uma base de $\Delta_q$) $f_i(q) \in \real$, $1 \leq i \leq k$, tais que
$q\circ V = \sum f_i(q) (q \circ X_i)$.

Como $V$ é suave e cada $X_i$ é suave, as funções $f_i:q \mapsto f_i(q)$ também são suaves.
Dessa forma $V|_U = \sum f_i X_i|_U$. Além disso, dadas quaisquer $k$ funções suaves
$g_1, \dots, g_k$, temos que $\sum g_i X_i|_U \in \X(\Delta)|_U$. Dessa forma,
$X_1, \dots, X_k$ são um conjunto gerador de $\X(\Delta)|_U$.

Como $\Lie \X(\Delta) = \X(\Delta)$ é localmente finitamente gerada, temos que
$T_p \mathcal{O}_p = \Lie_p \X(\Delta) = \Delta_p$, e portanto
$\mathcal{O}$ é uma folheação integral de $\Delta$. \qed

O Teorema de Frobenius não se sustenta quando $\Delta$ é uma distribuição singular:
a falha na demonstração acontece ao tentarmos achar as funções $f_i$ (é possível garantir
a existencia de tais funções mas não sua suavidade). Existem exemplos
de distribuições singulares satisfazendo a condição de Frobenius mas que não
são integráveis.

Dessa forma, para distribuições singulares, invocamos a descrição de
$T_p \mathcal{O}_p$ do Teorema da Órbita para fornecemos uma condição necessária
e suficiente para integrabilidade de distribuições.

Dado $P \in \Diff(M)$, denote por $\Delta_p \circ P = \{X_p \circ P : X_p \in \Delta_p\}$.

\begin{definition}
    Seja $X \in \X(M)$ e $\Delta$ uma distribuição singular. Então, dizemos
    que $\Delta$ é invariante com respeito a $X$ se,
    para todo $t \in \real$, $\Delta_p \circ e^{tX} \subseteq \Delta_{p \circ e^{tX}}$.
\end{definition}

\begin{theorem}[Teorema de Stefan-Sussmann]
    Seja $\Delta$ uma distribuição singular. Então, são equivalentes:

    \begin{itemize}
        \item $\Delta$ é invariante com respeito à todo $X \in \X(\Delta)$.
        \item $\Delta$ é integrável.
    \end{itemize}
\end{theorem}

\dem Primeiramente suponha que $\Delta$ é integrável, seja $L$ uma folheação integral
de $\Delta$ e $X \in \X(\Delta) = \X(L)$. Seja ainda $p \in M$. Como $e^{tX}$ mapeia
$L_p$ em $L_p$ (isto é, $e^{tX}(L_p) \subseteq L_p$), temos
que $d(e^{tX})_p (T_p L_p) \subseteq T_{e^{tX}(p)} L_p$, e portanto
$\Delta_p \circ e^{tX} \subseteq \Delta_{p \circ e^{tX}}$, ou seja,
$\Delta$ é invariante com respeito à $X$.

Agora suponha que $\Delta$ é invariante com respeito à todo $X \in \X(\Delta)$.
Sejam $X, Y \in \X(\Delta)$.
Como $p \circ (\Ad e^{tX}) Y = ((p \circ e^{tX}) \circ Y) \circ e^{-tX})$,
e $(p \circ e^{tX}) \circ Y \in \Delta_{p \circ e^{tX}}$, temos que
$p \circ (\Ad e^{tX}) Y \in \Delta_{p \circ e^{tX}\circ e^{-tX}}= \Delta_p$.

Considere a folheação $\mathcal{O}$ dada pelas órbitas de $\X(\Delta)$.
Como, para cada $p \in M$,

$$T_p \mathcal{O}_p = \{p \circ (\Ad e^{t_1 X_1} \circ \dots \circ e^{t_kX_k}) Y : X_1, \dots, X_k, Y \in \X(\Delta)\}$$

temos que $T_p \mathcal{O}_p \subseteq \Delta_p$. Como é claro que $\Delta_p \subseteq T_p \mathcal{O}_p$,
temos que $\mathcal{O}$ é uma folheação integral de $\Delta$. \qed

\part{Appendixes}
\label{apdx:EDOs}





%%%%%%%%%%%%%%
%%% bibliografia M. Alexandrino
%%%%%%%%%%%%% 
\begin{thebibliography}{99}


%\bibitem{AlexBettiol} M. M. Alexandrino,  R. G. Bettiol, 
%\emph{Lie Groups and Geometric Aspects of Isometric Actions}. Springer Verlag  (2015)  




\bibitem{Agrachev} A. Agrachev Y. Sachkov
\emph{Control theory from the Geometric viewpoint }
Encyclopedia of Mathematical Science, Control Theory and
Optimization.  

\bibitem{AlexBettiol} M. M. Alexandrino,  R. G. Bettiol, 
\emph{Lie Groups and Geometric Aspects of Isometric Actions}. Springer Verlag  (2015)  


\bibitem{mecanica-Holm} D.D. Holm, T. Schamah, C. Stoica
\emph{Geometric Mechanics and Symmetry}, Oxford Texts 
in Applied and Engineering Mathematics.

\bibitem{Lee} J. M. Lee, \emph{Introduction to smooth manifolds}, Springer 

\bibitem{Spivak} M. Spivak, 
\emph{A comprehensive introduction to differential geometry}, Perish or publish. 

%\bibitem{AlexBenignoHengameh}
%M. M. Alexandrino, B. O. Alves,
%H. R. Dehkordi, \emph{On Finsler transnormal functions}
%Differential Geometry and its Applications v 65 (2019) 93--107.


%\bibitem{AlexandrinoBettiol-livro}
%M. M. Alexandrino, R. G. Bettiol, \emph{Lie groups and geometric aspects of isometric actions},
%Springer Verlag (2015)

%\bibitem{AlexandrinoMestrado}
%M. M. Alexandrino, \emph{Hipersuperfícies de nível de uma função transnormal},
%dissertação de mestrado, PUC-Rio (1997)


%\bibitem{Alves-Phd} B.O Alves, \emph{Sobre folheações Finslerianas singulares}
%Tese de doutorado, IME-USP (2017)


%\bibitem{VideoGuilherme}
%G. Cerqueira Gonçalves, \emph{Algumas aplicações de mecânica na geometria Riemanniana e 
%Finsleriana} video baseado na apresentação do  28 Simpósio Internacional
%de Iniciação Científica e Tecnológica da USP: 
%\url{https://drive.google.com/file/d/1mSAdmCcIMt2Sk71LYsZNzwnrdBrYWjol/view?usp=sharing}
%\url{https://drive.google.com/file/d/1quQMnLn3AQAQR8ufPH8gBjHhHkSc-w4Y/view?usp=sharing }
%\url{https://drive.google.com/file/d/1quQMnLn3AQAQR8ufPH8gBjHhHkSc-w4Y/view?usp=sharing }


%\bibitem{HengamegSaa}
%H.R. 	Dehkordi  and S. Alberto, \textit{Huygens? envelope principle in Finsler spaces and analogue gravity.} Classical and Quantum Gravity 36.8 (2019): 085008.
	
%	\bibitem{SalomaoUmberto}
%U. L 	Hryniewicz  and P. A. Salomão, \emph{Introdu\c{c}\~{a}o a geometria Finsler.} IMPA, 2013.
	
%\bibitem{Markvorsen}
%S. Markvorsen,  \textit{A Finsler geodesic spray paradigm for wildfire spread modelling.} Nonlinear Analysis: Real World Applications 28 (2016): 208-228.

	


%\bibitem{Th} G. Thorbergsson,
%\textit{A survey on isoparametric hypersurfaces and their generalizations},
%Handbook of differential geometry, vol.\ 1, Elsevier Science, 2000.


%\bibitem{ThSurvey2} G. Thorbergsson, \textit{Transformation groups and submanifold geometry}, Rendiconti di Matematica, Serie VII, \textbf{25} (2005), 1--16.

%\bibitem{ThSurvey3} G. Thorbergsson, \textit{Singular Riemannian foliations and isoparametric submanifolds}, Milan J. Math. \textbf{78} (2010), 355--370.


		
	


\end{thebibliography}
%%%%%%%%%%%%
%%%%%%%%%%%%%%%%%%%%%%%%%%%%% bibliografia M Alexandrino
%%%%%%%%%%%

\end{document}

